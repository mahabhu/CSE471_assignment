\paragraph{Related Work.} A vast number of MARL algorithms have been proposed for iterative strategy adjustment while playing a game.  The most widely studied class of algorithms of this type involve each player running a no-regret algorithm on its own stream of rewards. The celebrated fictitious play algorithm \cite{brown1951iterative} and its descendants are special cases of this class. Although the convergence behavior of fictitious play and its variants has been studied extensively, convergence results are typically available only for games exhibiting special structural properties amenable to analysis \cite{hofbauer2002global,leslie2006generalised,baudin2022fictitious,sayin2022fictitious-zero,sayin2022fictitious-single}. %% 
Indeed, the convergence properties of fictitious play are intimately connected to those of \textit{best response dynamics}, a full information dynamical system evolving in continuous time where the evolution rule for player $i$'s strategy is governed by its best response multi-function. By harnessing such connections, convergence results for fictitious play and a number of other MARL algorithms have been obtained by analyzing the dynamical systems induced by specific update rules \cite{benaim2005stochastic,leslie2005individual,swenson2018best}.

A related line of research considers strategic dynamics defined by strategy update functions, taking the form $x^i_{t+1} = f^i ( \bx_t )$ in discrete time or an analogous form in continuous time. % 
In the case of deterministic strategy updates, \cite{hart2003uncoupled} studied strategic dynamics in continuous time and showed that if the strategy update functions, analogous to $f^i$ above, satisfy regularity conditions as well as a desirable property called uncoupledness, by which $f^i$ cannot depend on the reward functions of $i$'s counterplayers, then the resulting dynamics are not Nash convergent in general. These results were recently generalized by \cite{milionis2023impossibility}. Additional possibility and impossibility results were presented by \cite{babichenko2012completely}, who studied strategic dynamics in a different setting, where players do not observe counterplayer strategies. Under stochastic strategic dynamics, a number of positive results were obtained by incorporating exogenous randomness into one's strategy update, along with finite recall of recent play \cite{hart2006stochastic,foster2006regret,germano2007global}. In the regret testing algorithm of \cite{foster2006regret}, players revise their strategies according to whether or not their most recent strategy met a satisfaction criterion: if $x^i_t$ performed within $\epsilon$ of the optimal performance against $\bx^{-i}_t$, player $i$ continues using it and picks $x^i_{t+1} = x^i_t$. Otherwise, player $i$ experiments and selects $x^i_{t+1}$ according to a probability distribution over $\XX^i$. Conditional strategy updates similar to this have appeared in several other works, such as \cite{chien2011convergence,candogan2013near,chasparis2013aspiration}, and the regret testing algorithm has been extended in several ways \cite{germano2007global,arslan2017decentralized}. 


A game is said to have the \emph{satisficing paths property} if every initial strategy profile is connected to some equilibrium by a satisficing path. As we discuss in the next section, satisficing paths can be interpreted as a natural generalization of best response paths. Consequently, the problem of identifying games that have the satisficing paths property is a theoretically relevant question analogous to characterizing potential games \cite{monderer1996potential} or determining when a game has the fictitious play property \cite{monderer1996a,monderer1996fictitious}. The concept of satisficing paths was first formalized in \cite{yongacoglu2023satisficing} in the context of multi-state Markov games, where it was shown that $n$-player symmetric Markov games have the satisficing paths property and this fact could be used to produce a convergent MARL algorithm. However, the core idea of satisficing paths appeared earlier, before this formalization: in the convergence analysis of the regret testing algorithm in \cite{foster2006regret}, it was shown that two-player normal-form games have the satisficing paths property, though this terminology was not used. These earlier works made no claims about the existence of paths in general-sum $n$-player games, which is the focus of this paper. 